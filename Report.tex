\documentclass[12pt]{article}

\usepackage[a4paper, total={6in, 9in}]{geometry}

\author{Josiah Craw\\35046080}

\title{\huge{ENCE360 Assignment}}

\begin{document}

\maketitle

\newpage

\section{Algorithm Analysis}
\subsection{Describe the Algorithm}
The algorithm from lines 228-264 initially moves through each URL within the input file. It then
proceeds to get the URL from the file by replacing the end character with a null terminator. The
URL and the number of threads is then passed into get\_num\_tasks. This function destermins the
number of tasks required to complete the download given the number of threads. The byte size is
then returned. The number of tasks is used to loop through and add that number of tasks to the
todo task queue. This queue contains tasks which hold the URL and the min and max range for one
task. This separates the file into parts using the byte ranges. Next, the context containing the
todo and complete queues is passed into wait\_task. This function runs through all of the tasks in
the queue and processes them using http\_get\_content. The files are then merged and the file
fragments are deleted.

This algorithm is similar to the worker assignment algorithm that has been used in class.

\subsection{Improvements}
This algorithm could be improved by deleting the file fragments as they are merged into the final
file. Running through all the files twice is inefficient. Splitting the files within threads would
not necessarily increase efficiency however, this could help making the system if an internet
failure occurs. For this improvement to be effective on running the code would have to check if
the downloads had already occurred then catch up to the most recently downloaded file. Another
improvement could be merging and deleting the files in a thread while starting the next download
in the background. Alternatively the file fragments could be merged as they are downloaded
individually. 

\section{Performance}


\end{document}
